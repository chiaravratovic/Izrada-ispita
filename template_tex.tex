\documentclass{beamer}
\usepackage[utf8]{inputenc}
\usepackage[T1]{fontenc}
\usepackage[croatian]{babel}
\usepackage{url}


\title{Izrada ispita}
\author{Chiara Vratović, Karlo Krapić, Luka Skitarelić}
\date{23.1.2018}
\institute{Tehnički fakultet}

  \usetheme{Berlin}    
  \usecolortheme{whale}
  \usefonttheme{professionalfonts}  
  \setbeamertemplate{navigation symbols}{}
  \setbeamertemplate{caption}[numbered]


\begin{document}
	\frame {
		\titlepage 
	}
	\begin{frame}[fragile]
    	\frametitle{Test klasa dokumenta}
    	\begin{itemize} 
    		\item najjednostavniji način izrade ispita
    		\item dvije dodatne opcije
    		\item \begin{verbatim}\documentclass[addpoints][solution]{test}\end{verbatim}
    		\item addpoints
    				\begin{itemize} 
    					\item omogućuje zbrajanje bodova, izradu tablica bodova i ocjena
    					\item \begin{verbatim}\addpoints \noaddpoints\end{verbatim}
    					\item pale i gase addpoints opciju
    				\end{itemize}
    		\item answers 
    				\begin{itemize}
    					\item omogućuje ispis rješenja na pitanja  
    					\item \begin{verbatim}\printanswers\end{verbatim}
    					\item isti učinak ako se primijeni u preambuli dokumenta
    				\end{itemize}
    	\end{itemize}
    \end{frame}
    
    \begin{frame}[fragile]
    	\frametitle{Pitanja}
    	\begin{itemize}
    		\item \begin{verbatim}\begin{question}\end{question}\end{verbatim}
    		\item environment u koji stavljamo pitanja
    		\item \begin{verbatim}\question\end{verbatim}
    		\begin{itemize}
    			\item naredba koja započinje pitanje
    			\item automatski numerira pitanja
    		\end{itemize}
    		\item moguće ih je više puta podijeliti na dijelove i potpitanja 
    	\end{itemize}
    \end{frame}
    
    \begin{frame}[fragile]
    	\frametitle{Dijelovi}
    	\begin{itemize}
    		\item \begin{verbatim}\begin{parts,subparts,subsubparts}\end{verbatim}
    		\item za svaki postoji odgovarajuća naredba koja započinje pitanje
    		\item \begin{itemize}\begin{verbatim}\part \subpart \subsubpart\end{verbatim}\end{itemize}
    		\item moguće je urediti izgled brojeva i slova pri numeriranju pitanja i prikazu bodova
    	\end{itemize}	
    \end{frame}
    
    \begin{frame}
		\frametitle{Pitanja s više izbora}
		Postoje četiri okruženja koja možemo koristiti za generiranje pitanja s više izbora i sva se moraju započeti i završiti unutar question environmenta
			\begin{enumerate}
				\item{Choices okruženje}
				\item{Oneparchoices okruženje}
				\item{Checkboxes okruženje}
				\item{Oneparcheckboxes okruženje}
			\end{enumerate}
	\end{frame}
	
	\begin{frame}[fragile]
		\frametitle{Primjeri choices i oneparchoices okruženja}
		Choices okruženje koristimo kako bismo ponudili više izbora odgovora koji su postavljeni jedan ispod drugog
			\begin{itemize}
				\item
					\begin{verbatim}
					\begin{choices}
					\choice Prvi izbor
					\choice Drugi izbor
					\end{choices}
					\end{verbatim}
			\end{itemize}
		Oneparchoices okruženje koristimo kako bismo ponudili više izbora odgovora koji su postavljeni jedan za drugim
			\begin{itemize}
				\item
					\begin{verbatim}
					\begin{oneparchoices}
					\choice Biti
					\choice Ne biti
					\end{oneparchoices}
					\end{verbatim}
			\end{itemize}
	\end{frame}

	\begin{frame}[fragile]
		\frametitle{Primjeri checkboxes i oneparcheckboxes okruženja}
		Checkboxes okruženje koristimo kako bismo imali više izbora koji ispred sebe imaju praznu kućicu koju popunjavamo 
			\begin{itemize}
				\item
					\begin{verbatim}
					\begin{checkboxes}
					\choice Lijevo
					\choice Desno
					\end{checkboxes}
					\end{verbatim}
			\end{itemize}
		Oneparcheckboxes okruženje koristimo kako bismo imali više izbora koji ispred sebe imaju praznu kućicu koju popunjavamo i postavljeni su svi u jednom redu
			\begin{itemize}
				\item
					\begin{verbatim}
					\begin{oneparcheckboxes}
					\choice Buka
					\choice Tišina
					\end{oneparcheckboxes}
					\end{verbatim}
			\end{itemize}
	\end{frame}

	\begin{frame}[fragile]
    	\frametitle{Odgovori}
    	\begin{itemize}
    		\item Postoji pet vrsta prostora za odgovore
    		\begin{itemize}
    			\item prazan prostor \begin{verbatim}\vspace*{1in} ili \vspace*{\stretch{2}}\end{verbatim}
    			\item prazan pravokutnik \begin{verbatim}\makeemptybox{1in}\end{verbatim}
    			\item linije \begin{verbatim}\fillwithlines{1in}\end{verbatim}
    			\item isprekidane linije \begin{verbatim}\fillwithdottedlines{1in}\end{verbatim}
    			\item mreža \begin{verbatim}\fillwithgrid{1in}\end{verbatim}
    		\end{itemize}
    	\end{itemize}	
    \end{frame}

    \begin{frame}[fragile]
    	\frametitle{Odgovori} 
    	\begin{itemize}
    		\item \begin{verbatim}\vspace*{1in}\end{verbatim} 
    		ostavi prazan prostor nakon linije u kojoj se javlja
    		\item \begin{verbatim}\vspace*{\stretch{1}}\end{verbatim} 
    		podijeli prostor između pitanja u određenom omjeru
    		\item \begin{verbatim}\newpage\end{verbatim} 
    		upotrebom odmah nakon jedne od naredbi primjenjuje ju na cijelu stranicu
    		\item moguće je mijenjati boju, razmak između linija, širinu...
    	\end{itemize}	
    \end{frame}
    
    \begin{frame}[fragile] 
    	\frametitle{Odgovori}
    	\begin{itemize}
    		\item \begin{verbatim}\answerline\end{verbatim}
    			\begin{itemize}
    				\item koristi se za kratke odgovore
    				\item umetne prazan prostor i nakon toga liniju
    				\item kombinacija dviju naredbi
    			\end{itemize}
    		\item \begin{verbatim}\setlength\answerskip{2ex}\end{verbatim}
    			\begin{itemize}
    				\item naredba za veličinu praznog prostora
    			\end{itemize}
    		\item \begin{verbatim}\setlength\answerlinelength{1in}\end{verbatim}
    			\begin{itemize}
    				\item naredba za duljinu linije 
    			\end{itemize}
    	\end{itemize}
    \end{frame}

	\begin{frame}[fragile]
		\frametitle{Solution okruženje}
		\begin{itemize}
		\item Postoji 6 okruženja za zapisivanje rješenja
		\item  Zadane postavke su da se rješenja ne ispisuju pa o tome hoće li se isprintati rješenja odlučujemo s dvije naredbe:
			\begin{verbatim}
			\printanswers i \noprintanswers 
			\end{verbatim}
		\item Također postoji opcija za odgovore koju možemo dodati dokument klasi za ispite
			\begin{verbatim}
			\documentclass[answers]{exam}
			\end{verbatim}
		\end{itemize}
	\end{frame}

	\begin{frame}[fragile]
		\frametitle{Primjeri solution okruženja}
		Postoje 2 tipa okruženja 
		\begin{itemize}
		\item Solutionbox okruženje koje uvijek isprinta prazan kvadrat i rješenje unutar kvadrata ako se rješenja printaju 
		\item Solution, solutionorbox, solutionorlines, solutionordottedlines, solutionorgrid koji printaju rješenje ili ništa i jedina razlika među njima je količina prostora kojeg puštaju kad se 				rješenja ne printaju
		\end{itemize}
	\end{frame}
	
	\begin{frame}[fragile]
\frametitle{Bodovanje}
\emph{Broj bodova} za pojedini zadatak ispisujemo pored pitanja ili potpitanja u uglatim zagradama (u questions i parts okruženjima):
\begin{itemize}
	\item
		\begin{verbatim}
		\question[2] Što je Latex?
		\end{verbatim}
	\item
		\begin{verbatim}
		\question Zadana je matrica.
			\part[2] Izračunaj njenu determinantu.
			\part[3] Izračunaj njen inverz.
		\end{verbatim}
\end{itemize}
No, broj bodova ne mora uvijek biti cjelobrojni, možemo dodavati i polubodove tako da u opcije klase dokumenta uvedemo addpoints opciju, a polubodove definiramo naredbom 'half':
\begin{itemize}
	\item 
		\begin{verbatim}
		\documentclass[addpoints]{exam}
		...
		\question[2 \half] Riješi jednadžbu.
		\end{verbatim}
\end{itemize}
\end{frame}

\begin{frame}[fragile]
\frametitle{Bodovne tablice}
Razlikujemo \emph{3 tipa bodovnih tablica}:
\begin{itemize}
\item 
    \begin{verbatim}
	\gradetable[][] - ispisuje bodovnu tablicu za svako 
				  pitanje (bonus pitanja nisu uključena)
    \end{verbatim} 
\item 
    \begin{verbatim}
	\bonusgradetable[][] - ispisuje bodovnu tablicu samo 
					   za bonus pitanja
    \end{verbatim}
\item 
    \begin{verbatim}
	\combinedgradetable[][] - ispisuje bodovnu tablicu sa 
						  svim vrstama pitanja
    \end{verbatim}
\end{itemize}
\end{frame}	

\begin{frame}[fragile]
Tablice su zadane dvama parametrima koje smještamo u zagrade:
\begin{itemize}
\item prvim parametrom određujemo \emph{položaj tablice}:
	\begin{itemize}
	\item 
	\begin{verbatim}
	[h] - za horizontalnu tablicu
	\end{verbatim}
	\item 
	\begin{verbatim}
	[v] - za vertikalnu tablicu
	\end{verbatim}
	\end{itemize}

\item drugim parametrom određujemo \emph{kriterij ispisa bodova}:
	\begin{itemize}
	\item
		\begin{verbatim}
		[questions] - ispis bodova po pitanjima
		\end{verbatim}
	\item
		\begin{verbatim}
		[pages] - ispis bodova po stranicama
		\end{verbatim} 
	\end{itemize}
\end{itemize}
\end{frame}

\begin{frame}[fragile]
\frametitle{Izrada ispita na drugim jezicima}
Exam okružje podržano je samo na engleskom jeziku, no neke zadane riječi možemo prevesti i na drugi jezik tako da ih definiramo u preambuli, primjerice:
\begin{itemize}
\item
	\begin{verbatim}
	\pointpoints{bod}{bodovi} - riječ "point" mijenja u 
      "bod", a "points" u "bodovi"
	\end{verbatim}
\item
	\begin{verbatim}
	\bonuspointpoints{dodatni bod}{dodatni bodovi}
	\end{verbatim}
\item
	\begin{verbatim}
	\totalformat{Pitanje \thequestion: \totalpoints bodova} 
     - prikazuje format ukupnog izgleda "Pitanje x:
	       y bodova"
	\end{verbatim}
\end{itemize}
\end{frame}

\begin{frame}[fragile]
Isto to možemo učiniti sa zadanim riječima u bodovnim tablicama. Riječi u tablicama mijenjamo ovisno o:
\begin{enumerate}[1]
\item \emph{položaju}
	\begin{itemize}
	\item h - za horizontalnu
	\item v - za vertikalnu
	\end{itemize}

\item \emph{riječi}
	\begin{itemize}
	\item q - question
	\item pg - page
	\item p - points
	\item s - score
	\item t - total
	\end{itemize}
\end{enumerate}
\end{frame}

\begin{frame}[fragile]
\begin{enumerate}[3]
\item \emph{vrsti}
	\begin{itemize}
	\item 
		\begin{verbatim} 
		gradetable - (h/v) + (slovo riječi koju mijenjamo) 
		+ word + {nova riječ}, npr. hqword{Pitanje}
		\end{verbatim} 
\medskip
	\item 
		\begin{verbatim}
		bonusgradetable - b + (h/v) + (slovo riječi koju 
		mijenjamo) + word + {nova riječ}, npr. bvpgword{Stranica}
		\end{verbatim}
\medskip	
	\item 
		\begin{verbatim}
		combinedgradetable - c + (h/v) + (slovo riječi koju 
		mijenjamo) + word + {nova riječ}, npr. chpword{Bodovi}
		\end{verbatim}
	\end{itemize}

\end{enumerate}
\end{frame}

\begin{frame}[fragile]
\frametitle{Mathexam paket}
Ispit u Latexu možemo napraviti i na drugi način - korištenjem \emph{mathexam paketa}:
\begin{itemize}
	\item
		\begin{verbatim}
		\documentclass[11pt]{article}
		\usepackage{mathexam}
		\end{verbatim}
	\item
		mathexam paket automatski generira zaglavlja i podnožja za svaku stranicu osim ako ne definiramo drugačije pomoću \begin{verbatim} \usepackage[nohdr]{mathexam} \end{verbatim} 
\end{itemize}
\end{frame}

\begin{frame}[fragile]
\frametitle{Odgovori}
U mathexam paketu postoje razne naredbe za zapis odgovora i definiranja prostora za odgovor:
\begin{itemize}
	\item
		\begin{verbatim}
		\answer - umetne prostor i horizontalnu 
	          liniju za odgovor
		\end{verbatim}
	\item
		\begin{verbatim}
		\addanswer - umetne samo liniju za odgovor, ne i 
		             prostor
		\end{verbatim}
	\item
		\begin{verbatim}
		\(add)answer[veličina praznog prostora za odgovor, 
		             npr. 1cm ili 1in plus 1fill] - 
		             određuje točno koliko prostora treba 
		             pustiti za odgovor, inače je automatski
		\end{verbatim}
	\item	
		\begin{verbatim}
		\(add)answer*{$f'(x) =$} ili \(add)answer*{Odgovor:} - 
		     u zagrade stavljamo ono što želimo da piše umjesto 
		     'answer'
		\end{verbatim}
	\item
		\begin{verbatim}
		\noanswer - samo horizontalna crta
		\end{verbatim}
\end{itemize}

\end{frame}

\begin{frame}[fragile]
\frametitle{Zaglavlje}
Specifikacije koje se pišu u preambuli a definiraju \emph{zaglavlje} svake stranice ispita:
\begin{itemize}
	\item
		\begin{verbatim}
		\ExamName{Završni ispit}
		\end{verbatim}
	\item
		\begin{verbatim}	
		\ExamClass{Calculus III}
		\end{verbatim}
	\item
		\begin{verbatim}
		\ExamHead{\today}
		\end{verbatim}
\end{itemize}

Uz njih, u dokumentu možemo definirati i liniju za upis imena učenika te neke osnovne upute za pisanje testa:
\begin{itemize}
	\item
		\begin{verbatim}
		\ExamNameLine
		\end{verbatim}
	\item
		\begin{verbatim}
		\ExamInstrBox{Ovdje upisujemo osnovne upute.}
		\end{verbatim}
\end{itemize}

\end{frame}

\begin{frame}[fragile]
\frametitle{Literatura}
\bibliographystyle{IEEEtr}

\begin{thebibliography}{9}
\beamertemplatetextbibitems
\bibitem{hirschhorn}
    Philip Hirschhorn,
    \textit{Using the exam document class},
    Department of Mathematics
    Wellesley College, Wellesley,
    2017.
\bibitem{wikilink}
    Latex wiki. [Online] Dostupno na:
    https://en.wikibooks.org/wiki/LaTeX/Teacher%27s_Corner
\bibitem{hlavacek}
    Jan Hlavacek,
    \textit{The mathexam Package},
    2007.
\bibitem{sharelatexlink}
    Sharelatex tutorial. [Online] Dostupno na:
    \begin{verbatim}
https://www.sharelatex.com/learn/Typing_exams_in_LaTeX    
    \end{verbatim}

\end{thebibliography}
    
\end{frame}
\end{document}
